
\chapter{Background \& Objectives}

\section{Background}

Because database tables are hard to visualise and understanding the relationships between them is not trivial, this project is meant to make this process of design easier and more efficient. This application spots trivial mistakes like a table not having a primary key or tables with no foreign key constraints. It also validates the SQL that the user uploads and informs them of the syntax error that has been detected.

The main application of this project is to aid students in designing databases. Its aim is to find any errors or possible flaws in the structure and design of the tables before submitting their project. Another application of this project is to be used in marking student projects to quickly spot obvious mistakes and missing relationships between tables.

The background research consisted of reviewing any other projects that aimed at achieving a similar goal to this project and seeing their approach for inspiration. Mostly SQL parsing is done by either looking at every character in a string or using Regular Expressions \cite{Regex}, Based on this research, the decision was made to use a combination of both of these approaches to make sure the parsing was thorough but not tedious to write and maintain. Most of the current open-source SQL parsers were not fit for this application or lacked the functionality for parsing SQL Data Definition Language. This meant that the main focus on this project needed to be creating SQL parsing from scratch.

The main areas of research for this project are as follows:

\begin{itemize}
	\item JavaScript Framework
	\item CSS Framework
	\item Working Environment
	\item Server Hosting
	\item Version Control
\end{itemize}

\subsection{JavaScript Framework}

At the beginning of the project, a decision was made for the database visualiser to be a web-based project, which meant that there were choices to be made about the possible combinations of frameworks that could be used. For JavaScript, the decision was made to just use native JavaScript since the scripting that had to be done for this project didn't require any particular frameworks like React, Angular, or Vue.

\subsection{CSS Framework}

Since the main focus of this project is the functionality of the SQL parser, the styling of the webpage could be handled by a CSS framework. It was decided to use the newest version of Bootstrap \cite{Bootstrap} since it has components that can be used by default, which made the design aspect easier and faster. Tailwind was another option for the CSS framework, but it didn't provide these components, which is why Bootstrap was chosen.

\subsection{Working Environment}

To start working on the website, a local development server was created to provide a runtime environment. Node.js was used with Visual Studio Code \cite{Code} as the IDE. These were the main elements of the working environment.

\subsection{Server Hosting}

The Database Visualiser is meant to be used by many students at once, so it is meant to be made available on the internet, which meant that server hosting was needed. Free server hosting services\cite{Infinity} were used in the beginning, this was done to test that building the webpage locally worked when deployed to the server. Later on in the development process, the built webpage was deployed to the university's servers, which was a more reliable option since the demonstrations would take place on campus.

\subsection{Version Control}

Version control was used in this project to mainly back up the code base. Since only one developer was working on this project, it was not needed, but it worked well with the lifecycle model and process used. Due to its popularity in the industry, GitHub \cite{Github} was used.

\section{Analysis}

From the analysis of the task, it was clear that an SQL parser had to be written from scratch, which is the main task for this project. This meant that the input data had to be organized according to the SQL grammar so that it could be converted into a visual table and be interpreted by the user. The visualisation part of this project will consist of creating the tables on the webpage and drawing the relationships between the tables.

\subsection{Regular Expressions}

Regular Expressions are usually something that should not be used for parsing but it can be helpful \cite{Parsing}. If only Regular Expressions are used for parsing they can be very fragile and should only be used for simpler languages. This is why for this project Regular Expressions are used for only parts of the parsing and it does not make up the entire parser. This was an alternative approach that was looked into but it would not be appropriate for this project.

\subsection{Parser}

Regular expressions are usually something that should not be used for parsing, but they can be helpful \cite{Parsing}. If only regular expressions are used for parsing, they can be very fragile and should only be used for simpler languages. This is why for this project regular expressions are used for only parts of the parsing and not the entire parser. This was an alternative approach that was looked into, but it would not be appropriate for this project.

\subsection{Visualising} 

Visualising consists of displaying the tables after the SQL has been parsed, and if there are any relationships between tables, then those should be displayed as well. The way the tables are structured and laid out is also an important part of visualising since it should be a structure that is easily comprehensible by the user. It should also be displayed properly on the webpage in a responsive manner since the website can be used on devices of different sizes.

\subsection{Suggestion of Structure Faults}

After the SQL has been parsed and the database has been displayed to the user, the application should provide advice and possible faults of the database, if there are any. This should be clear and direct, and it also should provide feedback as to how to fix this issue. This would involve an analysis of the database structure to see if there are any keys missing or an incorrect design of the database.

\subsection{Scope}

To set an appropriate target for this project, it was decided to only create parsing for DDL (Data Definition Language) for PostgreSQL only. It was decided to only create parsing for DDL because it is the SQL commands that define a database schema, which deals with creating and modifying the structure of the database, which is what is going to be visualised. The structure of the database is what this project focuses on, and it is what will be validated for the user. Creating a parser for the entirety of PostgreSQL would be outside the scope of this project and would take up too much time from designing the visualising part of this project.

Another goal for this project was to create parsing for the DQL (Data Query Language). This would involve allowing the user to not only input their table structure but also their entries in the tables. This would allow the user to enter a query for the database, and this project would be able to parse the query and return the result of the query to the user. The user would then test for themselves to see if the returned entries were what they expected from their queries. However, this proved to be extra features that were not needed for the scope of this project, as it only mentioned identifying database design flaws and would prove to take too much time than the project allows.

%Include the auto-fixing of the database structure?

\section{Process}

For this project, a process adapted from Scrumban was used. The elements of Scrum that were incorporated were:  
\begin{itemize}
	\item Week-long sprints with weekly review and retrospective sessions for iterative planning at regular intervals.
	\item These meetings dictated how much work was pulled into the sprint based on complexity and priority of the work.
	\item Assured necessary levels of analysis and design before starting development.
\end{itemize}
The elements of Kanban that were incorporated were:
\begin{itemize}
	\item In addition to the weekly review and retrospective sessions, a short time was dedicated to process improvement.
	\item Kanban board to provide visualisation of the current items that were in progress, To-Do or in the Sprint Backlog.
\end{itemize}
